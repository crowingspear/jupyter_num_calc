\documentclass[12pt,a4j]{jarticle}
\usepackage[dvips]{graphicx}
\usepackage{amsmath,amsthm,amssymb}
%\topmargin -15mm\oddsidemargin -4mm\evensidemargin\oddsidemargin
%\textwidth 170mm\textheight 257mm\columnsep 7mm
\setlength{\fboxrule}{0.2ex}
\setlength{\fboxsep}{0.6ex}

\pagestyle{empty}
\begin{document}
\small{v.18.2} 
\hfill\small{2018/06/29 実施}
\begin{center}
{\gt\large{情報科学科 数式処理演習 最終個別 試験問題}}
\end{center}
\vspace{5mm}

以下の問題をpythonを用いて自力で解き,出力して提出せよ.80 点以下のメンバーがいるグループは来週補講.

\begin{enumerate}

\item 
\begin{enumerate}
\item 
(正規直交基底)

グラム・シュミットの直交化法により,次のベクトルから$\boldsymbol{R}^3$の正規直交基底を作れ.(15点)
\begin{equation*}
\textbf{\textit{x}}_1=(1,1,1),\ 
\textbf{\textit{x}}_2=(0,1,0),\ 
\textbf{\textit{x}}_3=(-1,1,0)
\end{equation*}

\item
(直交補空間)

  $V = \left\{(x_1, x_2, x_3) \in \boldsymbol{R}^3; 2x_1 - x_2 + x_3 =0, x_1 - 3x_2 + x_3 =0\right\}$の直交補空間$V^{\perp}$を求めよ.(15点)
\end{enumerate}

\item 
\begin{enumerate}
\item 
(Taylor展開)

次の関数を原点の周りで5次までTaylor展開せよ.また両関数をt=0..2でプロットせよ.(15点)
\begin{equation*}
v = \exp(-t)+1.0
\end{equation*}
\item
(積分の比較)

前問で扱った二つの関数をt=0..2で積分し結果を浮動小数点数で比較せよ.Taylor展開した関数の積分値の誤差を0.001以下にするには何次までの展開が必要か.(15点)

\end{enumerate}


\item 
座標平面上の放物線$y=1-x^2$を$C$とする.

$\frac{1}{2} < b\leqq 1$として,
放物線$C$上の2点Q(-1,0)とR($1-b, 2b-b^2$)を通る直線を$m$とする.
$m$の方程式は
\begin{equation*}
y = \fbox{ セ }\,x + \fbox{ ソ }
\end{equation*}
である.

$C$と直線$m$で囲まれた図形の面積$S_1$は
\begin{equation*}
S_1 = \frac{\fbox{ タチ }}{\fbox{ ツ }}\,b^3 + b^2-  \fbox{ テ }\,b+\frac{4}{3}
\end{equation*}
である.一方,$C$と直線$m$の$1-b \leqq x \leqq b$
の部分,および直線$x=b$で囲まれた図形の面積$S_2$は
\begin{equation*}
S_2 = \frac{\fbox{ ト }}{\fbox{ ナ }}\,b^3 + 2b^2-  
 \frac{\fbox{ ニ }}{\fbox{ ヌ }}\,b+\frac{2}{3}
\end{equation*}
である.よって,$S_1$と$S_2$の和$S$は
\begin{equation*}
S= S_1 + S_2 = 
\frac{\fbox{ ネ }}{\fbox{ ノ }}\,b^3 + 3b^2
- \frac{\fbox{ ハ }}{\fbox{ ヒ }}\,b+2
\end{equation*}
となる.

$\frac{1}{2} < b \leqq 1$のとき,$S$の増減を調べると,
$S$は$b=\sqrt{\fbox{ フ }} - \fbox{ ヘ }$で
最小値を取ることがわかる
(10点)

(2015年度大学入試センター試験 追試 数学II・B第2問(2))

\item 前問の放物線$C$の方程式を$y=1-0.5x^2$として問題を解け.
放物線$C$上の2点はQ($-\sqrt{2},0$)とR($\sqrt{2}-b,1-(\sqrt{2}-b)^2$)と読み替えよ.また,$S_2$を求めるときの範囲は$\sqrt{2}-b \leq x \leq b$と読み替えよ.また,数値解となるので,答えはかっこによらず小数点となる.(30点)

前問においても,
2点Q($x_1,y_1$)とR($x_2,y_2$)を通る直線の方程式は
$$
y-y_1 = \frac{y_1-y_2}{x_1-x_2}(x-x_1)
$$
を使うが,変数を一度個別に代入しておくのが得策.
\end{enumerate}


\end{document}